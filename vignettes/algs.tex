\documentclass{article}

\usepackage[T1]{fontenc} % Use 8-bit encoding that has 256 glyphs
\usepackage[english]{babel} % English language/hyphenation
\usepackage{amsmath, amsfonts, amsthm} % Math packages
\usepackage{cite}
%\usepackage[sort&compress,square,comma,authoryear]{natbib}

% makes color citations
%% \usepackage[
%%   %dvips,dvipdfm,
%%   colorlinks=true,urlcolor=blue,citecolor=red,linkcolor=red,bookmarks=true]{hyperref}

\usepackage{color}
\usepackage{pgfplots}
\usepackage{tikz}
%\pgfplotsset{compat=1.9} 
%\usepackage{hyperref}

\usepackage{algorithm}
\usepackage[noend]{algpseudocode}

\usepackage{graphicx}
%\usepackage{wrapfig}
\usepackage{paralist}
\usepackage{graphics} %% add this and next lines if pictures should be in esp format
\usepackage{epsfig} %For pictures: screened artwork should be set up with an 85 or 100 line screen

\usepackage{epstopdf} 
\usepackage[colorlinks=true]{hyperref}
\hypersetup{urlcolor=blue, citecolor=red}
%\usepackage{showkeys}

\newtheorem{theorem}{Theorem}[section]
\newtheorem{corollary}{Corollary}
\newtheorem*{main}{Main Theorem}
\newtheorem{lemma}[theorem]{Lemma}
\newtheorem{proposition}{Proposition}
\newtheorem{conjecture}{Conjecture}
\newtheorem*{problem}{Problem}
\theoremstyle{definition}
\newtheorem{definition}[theorem]{Definition}
\newtheorem{remark}{Remark}
\newtheorem*{notation}{Notation}
\newcommand{\ep}{\varepsilon}
\newcommand{\eps}[1]{{#1}_{\varepsilon}}
\newcommand{\bs}{\boldsymbol}
\allowdisplaybreaks[3]


% new commands %%%%%%%%%%%%%%%%%%%%%%%%%%%%%%%%%%%%%%%%%%%%%
\newcommand{\R}{\mathbb{R}}
\newcommand{\diag}{\text{diag}}
\DeclareMathOperator*{\argmin}{arg\,min}

\makeatletter
\def\BState{\State\hskip-\ALG@thistlm}
\makeatother

%\newcommand\numberthis{\addtocounter{equation}{1}\tag{\theequation}}
\begin{document}
\title{Canned Prediciton Algorithms}

\author{Yair Daon}

\maketitle

{\footnotesize 
 \centerline{Courant Institute, New York University}
   \centerline{New York, NY 10012, USA}
}

\bigskip
 
\begin{abstract}
  Pseudo-code for the simplex projection algorithm
  \cite{SugiharaMay90} and the S-map algorithm \cite{Sugihara94}.
  Algorithms are presented for the simple case of predicting one
  variable using its own time series.
\end{abstract}

\section{Notation}\label{section:notation}
\begin{itemize}
\item $E$ denotes the time lag (embedding dimension).
\item $k$ denotes the number of nearest neighbours we use. For
  the simplex method it is $k=E+1$ but for the S-map method it
  can be much larger.
\item $T_p$ denotes how many time-steps into the future we are trying
  to predict.
\item $X\in \R^n$ denotes a (potentially long) time series.
\item $y\in \R^{E}$ is a vector of lagged observations for which we
  want to make a prediciton --- $y_1$ is the current value, $y_2$ is
  the value one time step before and $y_{E}$ is the value $E-1$ time
  steps before now.
\item $\theta \geq 0$ is the tuning parameter in the S-map method.
\item $X_t^E = (X_t, X_{t-1},..., X_{t-E+1} )' \in \R^E$ denotes the
  lagged embedding vectors.
\item $\| v \|$ is an unspecified norm of $v$. We do not specify
  which norm to use and that choice is left to the user / reader.
\item $\| v \|_2^2 = \sum_i v_i^2$ is the squared 2-norm. 
\item Entries of matrices and vectors are indexed in the standard
  linear algebraic fashion, starting at $1$ (like the R standard) and
  not at $0$ (like the C/C++ and python standard).
\end{itemize}


\section{Helper Methods}

\subsection{Nearest Neighbours}\label{subsec:NN}
I will not write implementation of the nearst neighbours method, just
present its description. The method will be used with the signature
presented in algorithm \ref{alg:NN}.

The input variables $X,y$ and $k$ are defined in section
\ref{section:notation}. The method returns a list of indices $N =
\{N_1,...,N_k\}$ such that
\begin{equation*}
  \| X_{N_i}^{E} - y\| \leq \| X_{N_j}^{E} - y\|
  \text{ if } 1 \leq i \leq j \leq k,
\end{equation*}
%
\begin{algorithm}
  \caption{Find Nearest Neighbours}\label{alg:NN}
  \begin{algorithmic}[1]
    \Procedure{NearNeighbour}{$y, X, k$}
    \EndProcedure
  \end{algorithmic}
\end{algorithm}

\subsection{Least Squares}
A least squares method finds $x$ that minimizes the error in the
solution of an over-determined linear system (more equations than
variables). Below, $A \in \R^{p \times q},p > q$ and $b\in \R^p$ and
the least squares problem is to find
\begin{equation*}
  \hat{x} := \argmin_{x\in \R^q} \|Ax-b\|_2^2.
\end{equation*}
This problem can be solved using a Singular Value Decompostion (SVD),
as outlined in algorithm \ref{alg:SVDLS}.
\begin{algorithm}
  \caption{Least Squares via SVD}\label{alg:SVDLS}
  \begin{algorithmic}[1]
    \Procedure{LeastSquares}{$A,b$}
    \Comment{Assume $A\in \R^{p \times q}, p > q$.}
    \State $U, S, V \gets SVD(A)$ \Comment{Thus, $A = U S V'$}
    \State $S^{inv} \gets \textsc{zeros}( q,p )$
    \Comment{The zero matrix in $\R^{q \times p}$}
    \For{ $i =1,...,q$ }
    \If{ $S_{ii} > 10^{-5}S_{11}$ } \Comment{Note that $10^{-5}$ is arbitrary}
    \State $S^{inv}_{ii} \gets \frac{1}{S_{ii}}$
    \EndIf
    \EndFor
    \State $x \gets V S^{inv} U'b$
    \State \Return $x$
    \EndProcedure
  \end{algorithmic}
\end{algorithm}

\section{Simplex Projection}\label{section:simplex}
Ignoring ties in distances, minimal distances, minimal weights and
other potential hazards, the following algorithm performs Simplex
projection to predict $T_p$ time-steps ahead.
\begin{algorithm}
  \caption{Simplex Projection \cite{SugiharaMay90}}\label{alg:simplex}
  \begin{algorithmic}[1]
    \Procedure{SimplexPrediction}{$y, X, E, T_p$}

    \State $N \gets$ \textsc{NearNeighbour}($y, X, E+1$)
    \Comment{Find $E+1$ nearest neighbours.}
    \State $d \gets  \| X_{N_1}^{E} - y\|$ \Comment{Define the distance scale.}

    \For{$i=1,...,E+1$}  
    \State $w_i \gets \exp (-\| X_{N_i}^{E} - y\| / d )$
    \Comment{Compute weights.}
    \EndFor

    \State $\hat{y} \gets \sum_{i = 1}^{E+1} w_iX_{N_i+T_p} /
    \sum_{i = 1}^{E+1} w_i$ 
    \Comment{prediction = average of predicitions.}
    
    \State \Return $\hat{y}$
    \EndProcedure
  \end{algorithmic}
\end{algorithm}

\section{S-map}
Ignoring ties in distances, minimal distances, minimal weights and
other potential hazards, the following algorithm uses the S-map method
to predict $T_p$ time-steps ahead.
%
\begin{algorithm}
  \caption{S-map \cite{Sugihara94}}\label{alg:smap}
  \begin{algorithmic}[1]
    \Procedure{SmapPrediction}{$y, X, E, k, \theta, T_p$ }
    \State $N \gets$ \textsc{NearNeighbour}($y, X, k$)
    \Comment{Find NN to use for prediciton.}
    \State $d \gets \frac{1}{k} \sum_{i=1}^k \| X_{N_i}^{E} - y\|$
    \Comment{Sum of distances.}
    \For {$i=1,...,k$} 
    \State $w_i \gets \exp (-\theta \| X_{N_i}^{E} - y\| / d )$
    \Comment{Compute weights.}
    \EndFor
    \State $W \gets \diag(w_i)$ \Comment{Reweighting matrix.}
    \State $A \gets
    \begin{bmatrix}
      1          & X_{N_1} & X_{N_1- 1} & \dots  & X_{N_1 - E + 1} \\
      1          & X_{N_2} & X_{N_2- 1} & \dots  & X_{N_2 - E + 1} \\
      \vdots     & \vdots & \vdots   & \ddots & \vdots       \\
      1          & X_{N_k} & X_{N_k- 1} & \dots  & X_{N_k - E + 1} 
    \end{bmatrix} $
    \Comment{Design matrix.}
    
    \State $A \gets WA$ \Comment{Weighted design matrix.}
    \State $b \gets 
    \begin{bmatrix}
      X_{N_1 + T_p} \\
      X_{N_2 + T_p} \\
      \vdots  \\
      X_{N_k + T_p} 
    \end{bmatrix} $
    \Comment{Response vector.}
    \State $b \gets Wb$ \Comment{Weighted response vector.}
    \State $\hat{c} \gets \argmin_{c} \| Ac - b \|_2^2$
    \Comment{Least squares, can be solved via algorithm \ref{alg:SVDLS}.}
    \State $\hat{y} \gets \hat{c}_0 + \sum_{i=1}^E\hat{c}_iy_i$
    \Comment{Using the local linear model $\hat{c}$ for prediction.}
    \State \Return $\hat{y}$
    \EndProcedure
  \end{algorithmic}
\end{algorithm}

Note that $k$, the number of nearest neighbours used for prediciton,
can be very large compared to the embedding dimension $E$. Since $A
\in \R^{k \times (1+E)}$, this means that $A$ is ``tall and skinny''
and the system $Ac = b$ is \emph{over-determined} (it has more
equations than variables). This means (typically) that there does not
exist any $c$ that solves said system. This is why we seek a
least-squares solution instead.


\bibliographystyle{unsrt}
\bibliography{refs}

\end{document}
